\documentclass[11pt,oneside]{article}
\usepackage[utf8]{inputenc}  
\usepackage[T1]{fontenc}
\usepackage{amsmath,amssymb,bm}
\usepackage{titlesec}
\usepackage{titletoc}
\usepackage{graphicx}
\usepackage[margin=2.5cm]{geometry}
\usepackage[frenchb]{babel}
\usepackage{hyperref}
\AddThinSpaceBeforeFootnotes
\FrenchFootnotes
\usepackage{ae,lmodern}
\usepackage{sectsty}
\usepackage[usenames,dvipsnames]{xcolor}
\usepackage{enumitem}
\usepackage{caption}
\usepackage{multicol}
\usepackage{pifont}
\usepackage{fancyhdr}
\pagestyle{fancy}
\fancyhf{}
\rhead{Rapport}
\lhead{\leftmark}
\rfoot{Page \thepage}
\renewcommand{\headrulewidth}{2pt}
\renewcommand{\footrulewidth}{1pt}
\definecolor{Rapport}{RGB}{213,25,50}
\definecolor{airforceblue}{rgb}{0.36, 0.54, 0.66}
\definecolor{amber}{rgb}{1.0, 0.75, 0.0}
\sectionfont{\bfseries\color{Plum}\LARGE}
\subsectionfont{\color{Mulberry}}
\parskip=10pt
\setlength{\parindent}{1cm}
\usepackage{comment}
\begin{document}
\begin{titlepage}
\phantom{aaaaaaaaaaaaaaaaaaaaaaaaaaaaaaaaaaaaaaa
ytrfdytfugvghikuhjbiujbhaaaaaaaaaaaaaaa}


\begin{minipage}[c]{.46\linewidth}
		\centering
		\includegraphics[scale=0.1]{logo} 
	\end{minipage}
\hfill%
\begin{minipage}[c]{.46\linewidth}
		\centering
		\includegraphics[scale=0.04]{LE2P} 
\end{minipage}



\phantom{aaaaaaaaaaaaaaaaaaaaaaaaaaaaaaaaaaaaaaa
ytrfdytfugvghikuhjbiujbhaaaaaaaaaaaaaaa}
\center
\fbox{\begin{minipage}[t][1cm][c]{8cm}
\begin{center}
{\huge \bfseries \textcolor{Rapport}{Feuille de Route}}
\end{center}
\end{minipage}}\\[0.5cm]
\textbf{\Large \color{Mulberry} .}\\[0.5cm] 
\begin{minipage}{0.5\textwidth}
\begin{flushleft} \large
\hspace{0.22\textwidth}\emph{\underline{Auteurs}:}\\
\begin{multicols}{2}
\begin{itemize}[font=\color{airforceblue} \Large, label=\ding{47}, leftmargin=0cm]
\item{Hermanda \textsc{Tandrayen} \\ {\small{35008782}}}
\item{Sanjy \textsc{Maksim} \\ {\small{35001087}}}
\end{itemize}
\end{multicols}
\end{flushleft}
\end{minipage}
\begin{minipage}{0.45\textwidth}
\begin{flushright} \large
\emph{\underline{Enseignante}:}\phantom{aaaaa}\\
\begin{itemize}[font=\color{amber} \Large, label=\ding{80}, leftmargin=3.5cm]
\item{Beatrice \textsc{Morel}}
\end{itemize}
\end{flushright}
\end{minipage}\\[0cm]
\vspace{10cm} 
\begin{center}
2019/2020
\end{center}
\vfill
\end{titlepage}


\newpage
\part*{Objectifs de la semaine}
\begin{itemize}
	\item Trouver d'autres Data papers avec des sujets totalement différents afin d'avoir un template neutre
	\item Mettre en place un plan d'organisation de fichier commun
	\item Crée un Gantt
\end{itemize}



\part*{Taches effectuées}
\section*{Recherche d'autres datas papers}
\subsection*{Datas papers liés aux jeux vidéo}
\begin{flushleft}
Exploring the relationship between video game expertise and fluid intelligence
\end{flushleft}

\begin{flushleft}
Réf : https://www.ncbi.nlm.nih.gov/pmc/articles/PMC5687598/
\end{flushleft}

\begin{flushleft}
\textbf{Abstract:}
\end{flushleft}

-	On a des informations générales ;

\begin{flushleft}
Ex : Des centaines de millions de personnes jouent à des jeux vidéo intellectuels tous les jours.
\end{flushleft}

-	Une problématique ;

\begin{flushleft}
Ex : Des centaines de millions de personnes jouent à des jeux vidéo intellectuels tous les jours.
\end{flushleft}

-	Et on parle des études qui ont été réalisées dans le domaine concerné et qui seront détaillées dans l’article.

\begin{flushleft}
Ex : Ici, nous allons décrire deux études examinant un lien potentiel entre l’intelligence et la performance sur un des jeux les plus populaires dans son genre… Dans la première étude … Dans la deuxième étude …
\end{flushleft}

\newpage

\begin{flushleft}
\textbf{Introduction}
\end{flushleft}

-	Définitions ;

\begin{flushleft}
Ex : Les MOBAs sont des jeux d’action stratégique qui opposent spécifiquement deux équipes de 5 individus.
\end{flushleft}

-	Mise on contexte ;

\begin{flushleft}
Ex : Nous nous focalisons sur une performance dans une catégorie de jeu vidéo joué par des millions de personnes…
\end{flushleft}

\begin{flushleft}
\textbf{Etudes}
\end{flushleft}

-	Détails sur les différentes études présentées précédemment.

\begin{flushleft}
\textbf{Matériaux et méthodes}
\end{flushleft}

-	Présentations des matériaux avec leur fonctionnalité et explication de la démarche effectuée pour la réalisation de l’étude. 

\begin{flushleft}
\textbf{Résultats}
\end{flushleft}

-	Présentation des résultats (tableau, figures, etc..)

\begin{flushleft}
\textbf{Discussion}
\end{flushleft}

-	Discussion des résultats
-	Lien entre les résultats et la problématique


\begin{flushleft}
\textbf{Informations complémentaires}
\end{flushleft}

\begin{flushleft}
\textbf{Remerciements}
\end{flushleft}

\newpage

\begin{flushleft}
\textbf{Déclaration de financement}
\end{flushleft}

-	Partenariat

\begin{flushleft}
Ex: Ce travail a été supporté par ...
\end{flushleft}

\begin{flushleft}
\textbf{Disponibilité des données}
\end{flushleft}

-	Emplacement des données

\begin{flushleft}
Ex: DOI
\end{flushleft}

\begin{flushleft}
\textbf{Références}
\end{flushleft}

\newpage

\subsection*{Datas papers liés à ...}

...

\section*{Organisation de fichier commun}

- Mise en ordre de l'espace de travail partagé avec Gitkraken

\section*{Gantt}

- En cours de procédure

\newpage
\part*{Objectifs pour la semaine prochaine}
\begin{itemize}
	\item Définir un plan de travail efficace pour commencer un début de data paper
	

\end{itemize}


\end{document}