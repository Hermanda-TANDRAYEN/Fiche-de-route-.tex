\documentclass[11pt,oneside]{article}
\usepackage[utf8]{inputenc}  
\usepackage[T1]{fontenc}
\usepackage{amsmath,amssymb,bm}
\usepackage{titlesec}
\usepackage{titletoc}
\usepackage{graphicx}
\usepackage[margin=2.5cm]{geometry}
\usepackage[frenchb]{babel}
\usepackage{hyperref}
\hypersetup {colorlinks=true,linkcolor=blue,urlcolor=blue}
\AddThinSpaceBeforeFootnotes
\FrenchFootnotes
\usepackage{ae,lmodern}
\usepackage{sectsty}
\usepackage[usenames,dvipsnames]{xcolor}
\usepackage{enumitem}
\usepackage{caption}
\usepackage{multicol}
\usepackage{pifont}
\usepackage{fancyhdr}
\pagestyle{fancy}
\fancyhf{}
\rhead{Rapport}
\lhead{\leftmark}
\rfoot{Page \thepage}
\renewcommand{\headrulewidth}{2pt}
\renewcommand{\footrulewidth}{1pt}
\definecolor{Rapport}{RGB}{213,25,50}
\definecolor{airforceblue}{rgb}{0.36, 0.54, 0.66}
\definecolor{amber}{rgb}{1.0, 0.75, 0.0}
\sectionfont{\bfseries\color{Plum}\LARGE}
\subsectionfont{\color{Mulberry}}
\parskip=10pt
\setlength{\parindent}{1cm}
\begin{document}
\begin{titlepage}
\phantom{aaaaaaaaaaaaaaaaaaaaaaaaaaaaaaaaaaaaaaa
ytrfdytfugvghikuhjbiujbhaaaaaaaaaaaaaaa}


\begin{minipage}[c]{.46\linewidth}
		\centering
		\includegraphics[scale=0.1]{logo} 
	\end{minipage}
\hfill%
\begin{minipage}[c]{.46\linewidth}
		\centering
		\includegraphics[scale=0.04]{LE2P} 
\end{minipage}



\phantom{aaaaaaaaaaaaaaaaaaaaaaaaaaaaaaaaaaaaaaa
ytrfdytfugvghikuhjbiujbhaaaaaaaaaaaaaaa}
\center
\fbox{\begin{minipage}[t][1cm][c]{8cm}
\begin{center}
{\huge \bfseries \textcolor{Rapport}{Feuille de Route}}
\end{center}
\end{minipage}}\\[0.5cm]
\textbf{\Large \color{Mulberry} .}\\[0.5cm] 
\begin{minipage}{0.5\textwidth}
\begin{flushleft} \large
\hspace{0.22\textwidth}\emph{\underline{Auteurs}:}\\
\begin{multicols}{2}
\begin{itemize}[font=\color{airforceblue} \Large, label=\ding{47}, leftmargin=0cm]
\item{Hermanda \textsc{Tandrayen} \\ {\small{35008782}}}
\item{Sanjy \textsc{Maksim} \\ {\small{35001087}}}
\end{itemize}
\end{multicols}
\end{flushleft}
\end{minipage}
\begin{minipage}{0.45\textwidth}
\begin{flushright} \large
\emph{\underline{Enseignante}:}\phantom{aaaaa}\\
\begin{itemize}[font=\color{amber} \Large, label=\ding{80}, leftmargin=3.5cm]
\item{Beatrice \textsc{Morel}}
\end{itemize}
\end{flushright}
\end{minipage}\\[0cm]
\vspace{10cm} 
\begin{center}
2019/2020
\end{center}
\vfill
\end{titlepage}


\newpage
\part*{Objectifs de la semaine}
\begin{itemize}
	\item Trouver d'autres Data papers avec des sujets totalement différents afin d'avoir un template neutre
	\item Mettre en place un plan d'organisation de fichier commun
\end{itemize}



\part*{Taches effectuées}
\section*{Plan d'organisation des fichiers commun}

\underline{Organisation du dossier: Fiche-de-route-.tex}
\begin{itemize} [font=\color{airforceblue} \Large, label=\ding{217}, leftmargin=0cm] 
\item  Doc exploitable:	
	\begin{itemize} 
	\item Document texte:
		\begin{description}
		\item[fichier texte word, bloc note, tableur excel ect.]
		\end{description}
	\item Document image:
		\begin{description}
		\item[fihier Pdf, png, Jpeg, réutilisable en figure.]
		\end{description}
	\end{itemize}

\item Feuille de route:
	\begin{itemize}
	\item  Feuille de route n: 
		
		\begin{description}
		\item[  weekly n , en pdf et .tex (obligatoire), 					et 				images associées au document LateX ]
		\end{description}
	
	\end{itemize}
\end{itemize}
Le plan reste à être optimisé il nous reste à comprendre comment 
partitionner le document latex, et comment faire 
appel aux images par un chemin relatif et non absolu.
\newpage
\section*{Observation des similitudes sur les datas papers}
Cette fois-ci afin d'avoir une réelle vue d'ensemble de ce qui peut ce faire comme DP \footnote{Data Paper}, nous avons aussi voulu consulter des sujets
 différents de la météorologie, (sans le mettre de côté).\\
Nous avons gardé les journaux les plus pertinents et avec les IF \footnote{Impact Factor} les plus élever:\\
Atomic Data and Nuclear Data Tables IF 6.349\\
Progrès dans les ressources en eau  IF 3.673\\
Agricultural and Forest Meteorology IF 4.189\\

\subsection*{Datas papers liés aux jeux vidéo}
\begin{flushleft}
Exploring the relationship between video game expertise and fluid intelligence
\end{flushleft}

\begin{flushleft}
Réf : https://www.ncbi.nlm.nih.gov/pmc/articles/PMC5687598/
\end{flushleft}

\begin{flushleft}
\textbf{Abstract:}
\end{flushleft}

-	On a des informations générales ;

\begin{flushleft}
Ex : Des centaines de millions de personnes jouent à des jeux vidéo intellectuels tous les jours.
\end{flushleft}

-	Une problématique ;

\begin{flushleft}
Ex : Des centaines de millions de personnes jouent à des jeux vidéo intellectuels tous les jours.
\end{flushleft}

-	Et on parle des études qui ont été réalisées dans le domaine concerné et qui seront détaillées dans l’article.

\begin{flushleft}
Ex : Ici, nous allons décrire deux études examinant un lien potentiel entre l’intelligence et la performance sur un des jeux les plus populaires dans son genre… Dans la première étude … Dans la deuxième étude …
\end{flushleft}

\newpage

\begin{flushleft}
\textbf{Introduction}
\end{flushleft}

-	Définitions ;

\begin{flushleft}
Ex : Les MOBAs sont des jeux d’action stratégique qui opposent spécifiquement deux équipes de 5 individus.
\end{flushleft}

-	Mise on contexte ;

\begin{flushleft}
Ex : Nous nous focalisons sur une performance dans une catégorie de jeu vidéo joué par des millions de personnes…
\end{flushleft}

\begin{flushleft}
\textbf{Etudes}
\end{flushleft}

-	Détails sur les différentes études présentées précédemment.

\begin{flushleft}
\textbf{Matériaux et méthodes}
\end{flushleft}

-	Présentations des matériaux avec leur fonctionnalité et explication de la démarche effectuée pour la réalisation de l’étude. 

\begin{flushleft}
\textbf{Résultats}
\end{flushleft}

-	Présentation des résultats (tableau, figures, etc..)

\begin{flushleft}
\textbf{Discussion}
\end{flushleft}

-	Discussion des résultats
-	Lien entre les résultats et la problématique


\begin{flushleft}
\textbf{Informations complémentaires}
\end{flushleft}

\begin{flushleft}
\textbf{Remerciements}
\end{flushleft}

\newpage

\begin{flushleft}
\textbf{Déclaration de financement}
\end{flushleft}

-	Partenariat

\begin{flushleft}
Ex: Ce travail a été supporté par ...
\end{flushleft}

\begin{flushleft}
\textbf{Disponibilité des données}
\end{flushleft}

-	Emplacement des données

\begin{flushleft}
Ex: DOI
\end{flushleft}

\begin{flushleft}
\textbf{Références}
\end{flushleft}

\newpage

 
\twocolumn
\setlength{\columnseprule}{0.01cm}
\begin{center}
Données atomiques et intensités de ligne pour l'ion SV
\\
-Data Journal: Atomic Data and Nuclear Data Tables-\\
\url {https://doi.org/10.1016/j.adt.2016.06.002}
\end{center}

\noindent Outline:
\begin{itemize}
\item Abstrait
\item Mots clés
\item 1 . introduction
\item 1 . introduction
\item 3 . Évaluation des résultats
\item ...
\item 6 . Conclusions
\item Reconnaissance
\item Références
\item Explication des tableaux
\end{itemize}

\line (1,0){200}

\begin{center}
Évaluation des variations et des tendances de la sécheresse saisonnière en Europe centrale\\
-Data Journal: Progrès dans les ressources en eau-\\
\url {https://doi.org/10.1016/j.advwatres.2019.03.005}
\end{center}
\noindent Outline:
\begin{itemize}
\item Points forts
\item Abstrait
\item Résumé graphique
\item Mots clés 
\item 1 . introduction  
\item 2 . Données et méthodes. 
\item 3 . Résultats
\item 4 . Discussion
\item 5 . Conclusions
\item Remerciements
\item Annexe . Matériel supplémentaire
\item Références
\end{itemize}

\begin{center}
Phenology acts as a primary control of urban vegetation cooling and warming: A synthetic analysis of global site observations
\\
-Data Journal: Agricultural and Forest Meteorology-\\
\url {https://doi.org/10.1016/j.agrformet.2019.107765}
\end{center}

\noindent Outline:
\begin{itemize}
\item Points forts
\item Abstrait
\item Mots clés
\item Les acronymes
\item 1 . introduction
\item 2 . matériaux et méthodes
\item 3 . résultats et discussion
\item 4 . Conclusions
\item Annexe . Matériel supplémentaire
\item Données de recherche
\item Références
\end{itemize}

Nous avons ici, comme pour les recherches précédentes les mêmes parties qui reviennent.
\newpage
\onecolumn
Un journal m'a beaucoup plus interpelé dans le sens où nous avons ce qu'il faut pour présenter les données du projet, le voici:

\begin{center}
Évaluations des estimations de la productivité primaire brute à l'aide de modèles basés sur les données satellitaires utilisant des sites d'observation de covariance de tourbillons dans l'hémisphère nord
\\
-Data Journal: Agricultural and Forest Meteorology-\\
\url {https://doi.org/10.1016/j.agrformet.2019.107771}
\end{center}

\noindent Outline:
\begin{itemize}
\item Points forts
\item Abstrait
\item Mots clés
\item 1 . introduction
\item 2 . Données et sites
\item 3 . Méthodologie
\item 4 . Résultats
\item 5 . Discussion
\item 6 . Conclusion
\item Déclaration d'intérêts concurrents
\item Remerciements
\item Annexe . Matériel supplémentaire
\item Références
\end{itemize}

\part*{Objectifs pour la semaine prochaine}
\begin{itemize}
	\item En attente.
	

\end{itemize}

\end{document}