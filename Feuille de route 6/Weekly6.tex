\documentclass[11pt,oneside]{article}
\usepackage[utf8]{inputenc}  
\usepackage[T1]{fontenc}
\usepackage{amsmath,amssymb,bm}
\usepackage{titlesec}
\usepackage{titletoc}
\usepackage{graphicx}
\usepackage[margin=2.5cm]{geometry}
\usepackage[frenchb]{babel}
\usepackage{hyperref}
\hypersetup {colorlinks=true,linkcolor=blue,urlcolor=blue}
\AddThinSpaceBeforeFootnotes
\FrenchFootnotes
\usepackage{ae,lmodern}
\usepackage{sectsty}
\usepackage[usenames,dvipsnames]{xcolor}
\usepackage{enumitem}
\usepackage{caption}
\usepackage{multicol}
\usepackage{pifont}
\usepackage{fancyhdr}
\pagestyle{fancy}
\fancyhf{}
\rhead{Rapport}
\lhead{\leftmark}
\rfoot{Page \thepage}
\renewcommand{\headrulewidth}{2pt}
\renewcommand{\footrulewidth}{1pt}
\definecolor{Rapport}{RGB}{213,25,50}
\definecolor{airforceblue}{rgb}{0.36, 0.54, 0.66}
\definecolor{amber}{rgb}{1.0, 0.75, 0.0}
\sectionfont{\bfseries\color{Plum}\LARGE}
\subsectionfont{\color{Mulberry}}
\parskip=10pt
\setlength{\parindent}{1cm}
\begin{document}
\begin{titlepage}
\phantom{aaaaaaaaaaaaaaaaaaaaaaaaaaaaaaaaaaaaaaa
ytrfdytfugvghikuhjbiujbhaaaaaaaaaaaaaaa}


\begin{minipage}[c]{.46\linewidth}
		\centering
		\includegraphics[scale=0.1]{logo} 
	\end{minipage}
\hfill%
\begin{minipage}[c]{.46\linewidth}
		\centering
		\includegraphics[scale=0.04]{LE2P} 
\end{minipage}



\phantom{aaaaaaaaaaaaaaaaaaaaaaaaaaaaaaaaaaaaaaa
ytrfdytfugvghikuhjbiujbhaaaaaaaaaaaaaaa}
\center
\fbox{\begin{minipage}[t][1cm][c]{8cm}
\begin{center}
{\huge \bfseries \textcolor{Rapport}{Feuille de Route}}
\end{center}
\end{minipage}}\\[0.5cm]
\textbf{\Large \color{Mulberry} .}\\[0.5cm] 
\begin{minipage}{0.5\textwidth}
\begin{flushleft} \large
\hspace{0.22\textwidth}\emph{\underline{Auteurs}:}\\
\begin{multicols}{2}
\begin{itemize}[font=\color{airforceblue} \Large, label=\ding{47}, leftmargin=0cm]
\item{Hermanda \textsc{Tandrayen} \\ {\small{35008782}}}
\item{Sanjy \textsc{Maksim} \\ {\small{35001087}}}
\end{itemize}
\end{multicols}
\end{flushleft}
\end{minipage}
\begin{minipage}{0.45\textwidth}
\begin{flushright} \large
\emph{\underline{Enseignants}:}\phantom{aaaaa}\\

\begin{itemize}[font=\color{amber} \Large, label=\ding{80}, leftmargin=3.5cm]
\item{Beatrice \textsc{Morel}}
\item{Alexandre \textsc{Graillet}}
\item{Mathieu \textsc{Delsaut}}
\end{itemize}

\end{flushright}
\end{minipage}\\[0cm]
\vspace{10cm} 
\begin{center}
2019/2020
\end{center}
\vfill
\end{titlepage}


\newpage
\part*{Objectifs de la semaine}
\begin{itemize}
	\item Trouver des revues permettant de publier des data papers
\end{itemize}



\part*{Taches effectuées}
\section*{Recherche de revue}
\subsection*{Exemples de revue (+ licence et coût de publication)}

\begin{flushleft}
\url{Réf : https://coop-ist.cirad.fr/content/download/6265/45560/version/3/file/Coopist-Revues+publiant+des+datapapers-nov+2017.pdf}\hypersetup {colorlinks=true,linkcolor=blue,urlcolor=blue}
\end{flushleft}

\begin{flushleft}
\textbf{BMC Ecology (BioMed Central):}
\end{flushleft}

-	Revue en libre accès : coût de publication (2017): 1745 Euros.

-	Facteur d'impact en 2016: 2.896

\begin{flushleft}
\textbf{Ecological Research – Ecological Research Archives (Springer) }
\end{flushleft}

-	Pas de coût de publication. Les data papers sont en accès libre.

-	Facteur d'impact en 2016: 1.283

\begin{flushleft}
\textbf{Ecology - Ecological Archives (Ecological Society of America / Wiley): }
\end{flushleft}

-	Coût de publication (2017): 250 dollars. Les data papers
sont en accès libre.

-	Facteur d'impact en 2016: 4.809

\begin{flushleft}
\textbf{Global Ecology and Biogeography (Wiley)}
\end{flushleft}

-	Pas de coût de publication, la revue n’est pas en libre accès. Coût de l’option Open access
(2017) : 3500 Euros.

-	Facteur d'impact en 2016: 6.045

\newpage

\subsection*{La Cirad}

\begin{flushleft}
La base \textbf{Où publier ?} du Cirad regroupe près de 2000 revues dont celles présentées précédemment.
\end{flushleft}


\begin{flushleft}
\url{ Ref: http://ou-publier.cirad.fr/index.php}
\end{flushleft}

Elle propose une sélection de revues dans des thèmes définis, répondant à des critères de qualité et est régulièrement mise à jour.

\newpage
\part*{Objectifs pour la semaine prochaine}
\begin{itemize}
	\item En savoir un peu plus sur les data papers
\end{itemize}


\end{document}